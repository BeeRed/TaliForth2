Tali Forth was released in the public domain with a happily overcommentated
source code to make it easy for other people to understand and adapt the code
for their own uses.

\section{Adding new words}

The easiest way to add new words to Tali Forth is to include them in the file
\texttt{forth\_code/user\_words.fs}. 


\section{Deeper changes}

Tali Forth was not only placed in the public domain to honor the tradition of
giving the code away freely. It is also to let people play around with it and
adapt it to their own machines. This is also the reason it is (perversely)
overcommented.

To work on the internals of Tali Forth, you will need the Ophis\index{Ophis
assembler|textbf} assembler.

\subsection{The Ophis Assembler}

Michael Martin's\index{Martin, Michael} Ophis Cross-Assember can be downloaded from
\href{http://michaelcmartin.github.io/Ophis/}{http://michaelcmartin.github.io/Ophis/}.
It uses a slightly different format than other assemblers, but is in
Python\index{Python} and
therefore will run on almost any operating system. To install Ophis on Windows,
use the link provided above. For Linux\index{Linux}:

\begin{lstlisting}[frame=single]
        git clone https://github.com/michaelcmartin/Ophis
        cd src
        sudo python setup.py install
\end{lstlisting}

Switch to the folder where the Tali code lives, and assemble with the primitive
shell script provided: \texttt{./assemble.sh} The script also automatically
updates the file listings in the \texttt{docs} folder. Note that Ophis will not
accept math operation characters in label names because it will try to perform
those operations. Because of this, we use underscores for label names.

\subsection{General notes}

\begin{itemize}

        \item The X register\index{X register} should not be changed without
                saving its pointer status.

        \item The Y register\index{Y register} is free to be changed by
                subroutines. This means it should not be expected to survive
                subroutines unchanged.

        \item All words should have one point of entry -- the \texttt{xt\_word}
                link -- and one point of exit at \texttt{z\_word}. In may cases,
                this means a branch to an internal label \texttt{done} right
                before \texttt{z\_word}.

        \item Because of the way native compiling works, the usual trick of
                combining \texttt{jsr/rts} pairs to a single \texttt{jmp}
                (usually) doesn't work.

\end{itemize}


\subsection{Coding style}\index{style, coding}

Until I get around to writing a tool for Ophis\index{Ophis assembler} assembler
code that formats the source file the way gofmt does for Go\index{Go} (golang),
I work with the following rules:

\begin{itemize}

        \item Actual opcodes are indented by \textbf{two tabs}

        \item Tabs are \textbf{eight characters long} and converted to spaces

        \item Function-like routines are followed by a one-tab indented
                `function doc' based on the Python 3\index{Python} model: Three
                quotation marks at the start, three at the end it its own line,
                unless it is a one-liner. This should make it easier to
                automatically extract the docs for them at some point.

        \item The native words have a special commentary format that allows the
                automatic generation of word list by a tool in the tools folder,
                see there for details.

        \item Assembler mnenomics are lower case. I get enough uppercase
                insanity writing German, thank you very much.

        \item Hex numbers are also lower case, such as \texttt{\$FFFE}

        \item Numbers in mnemonics are a stripped-down as possible to reduce
                visual clutter: \texttt{lda 0,x} instead of \texttt{lda \$00,x}.

        \item Comments are included like popcorn to help readers who are new
                both to Forth and 6502 assembler.

\end{itemize}


\section{Testing}\index{testing}

There is no automatic or formal test suite available at this time, and due to
space considerations, there probably never will be. This manual includes a list
of test cases that can be applied by hand in the appendix.


\section{Code Cheat Sheet}\index{cheat sheet, code}

\subsection{The Stack Drawing}\index{stack}
This is your friend and should probably go on your wall or something.

\begin{lstlisting}[frame=single]
                +--------------+
                |          ... |
                +-            -+
                |              |   ...
                +-  (empty)   -+
                |              |  FE,X
                +-            -+
          ...   |              |  FF,X
                +==============+
         $0076  |           LSB|  00,X   <-- DSP (X Register)
                +-    TOS     -+
         $0077  |           MSB|  01,X
                +==============+
         $0078  |  (garbage)   |  02,X   <-- DSP0
                +--------------+
         $0079  |              |  03,X
                + (floodplain) +
         $007A  |              |  04,X
                +--------------+
\end{lstlisting}

\subsection{Coding idioms}
While coding a Forth, there are certain assembler fragments that get repeated
over and over again. These could be included as macros, but that can make the
code harder to read for somebody only familiar with basic assembly.

Some of these fragments could be written in other variants, such as the `push
value' version, which could increment the DSP twice before storing a value. We
try to keep these in the same sequence (a "dialect" or "code mannerism" if you
will) so we have the option of adding code analysis tools later.

\begin{description}

        \item [\texttt{drop}] cell of top of the Data Stack 

                \begin{lstlisting}[frame=single]
                inx
                inx
                \end{lstlisting}

        \item [\texttt{push}] a value to the Data Stack.  Remember the Data Stack
                Pointer (DSP, the X register of the 65c02) points to the LSB of
                the TOS value.

                \begin{lstlisting}[frame=single]
                dex
                dex
                lda $<LSB>      ; or pla, jsr kernel_getc, etc.
                sta 0,x
                lda $<LSB>      ; or pla, jsr kernel_getc, etc.
                sta 1,x
                \end{lstlisting}

        \item [\texttt{pop}] a value off the Data Stack

                \begin{lstlisting}[frame=single]
                lda 0,x
                sta $<LSB>      ; or pha, jsr kernel_putc, etc
                lda 1,x
                sta $<MSB>      ; or pha, jsr kernel_putc, etc
                inx
                inx
                \end{lstlisting}

\end{description}

\subsection{vi shortcuts}

One option for these is to add abbreviations to your favorite editor, which
should of course be vim, because vim is cool. There are examples for that
further down. They all assume that auto-indent is on and we are two tabs in with
the code, and use \texttt{\#} at the end of the abbreviation to keep them
separate from the normal words. My \texttt{\textasciitilde/.vimrc} file contains the following
lines for work on \texttt{.asm} files:
 

\begin{lstlisting}[frame=single]
 ab drop# inx<tab><tab>; drop<cr>inx<cr><left>
 ab push# dex<tab><tab>; push<cr>dex<cr>lda $<LSB><cr>sta $00,x<cr>lda $<MSB><cr>sta $01,x<cr><up><up><up><up><end>
 ab pop# lda $00,x<tab><tab>; pop<cr>sta $<LSB><cr>lda $01,x<cr>sta $<MSB><cr>inx<cr>inx<cr><up><up><up><up><up><end>
\end{lstlisting}
 
