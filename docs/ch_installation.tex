\section{Downloading}

Tali Forth was created to be easy to get started with. In fact, all you should
need is the \texttt{ophis.bin} binary file and the
\texttt{py65mon}\index{py65mon@\texttt{py65mon}} simulator.

\subsection{Downloading Tali Forth}

The newest version of Tali Forth 2 lives on GitHub\index{GitHub} at
\href{https://github.com/scotws/TaliForth2}{https://github.com/scotws/TaliForth2}.
You can either clone the code with \texttt{git}\index{git@\texttt{git}} or
simply download it. To just try the program, all you need is the
\texttt{ophis.bin} binary. 

\subsection{Downloading the py65mon Simulator}\index{py65mon|textbf}

Tali was written to run out of the box on the 
\texttt{py65mon} simulator from
\href{https://github.com/mnaberez/py65}{https://github.com/mnaberez/py65}. This
is a Python\index{Python} program that should run on various operating systems. 

To install py65mon on Linux\index{Linux}, use the command \texttt{sudo pip
install -U py65}. If you don't have PIP\index{PIP} installed, you will have to
add it first with \texttt{sudo apt-get install python-pip}.  There is a
\texttt{setup.py} script as part of the package.

\section{Running the binary}

To start the emulator, run:
\begin{lstlisting}[frame=lines]
        py65mon -m 65c02 -r ophis.bin
\end{lstlisting}

\noindent Note that the option \texttt{-m 65c02} is required, because Tali Forth
makes extensive use of the additional commands of the CMOS version and will not
run on a stock 6502 MPU.

