
% ---------------------------------------
\section{Stack}

Tali Forth 2 uses the lowest part of the top half of Zero Page for the Data
Stack (DS). This leaves the lower half of the Zero Page for any kernel stuff the
user might require. The DS therefore grows towards the initial user variables.
See the file \texttt{definitions.asm} for details.  Because of the danger of
underflow, it is recommended that the user kernel's variables are keep closer to
\texttt{\$0100} than to \texttt{\$007f}.

The X register is used as the Data Stack Pointer (DSP). It points to the least
significant byte of the current top element of the stack (`Top of the Stack',
TOS).\footnote{In the first versions of Tali, the DSP pointed to the next
\textit{free} element of the stack. The new system makes detecting underflow
easier and parallels the structure of Liara Forth.}

Initially, the DSP points to \texttt{\$78}, not \texttt{\$7F} as might be
expected. This provides a few bytes as a `floodplain' in case of underflow. The
initial value of the DSP is defined as \texttt{dsp0} in the code.

\subsection{Single cell values} Since the cell size is 16 bits, each stack entry
consists of two bytes. They are stored little endian (least significant byte
first). Therefore, the DSP points to the LSB of the current TOS.\footnote{Try
reading that last sentence to a friend who isn't into computers. Aren't
abbreviations fun?}

Because the DSP points to the current top of the stack, the byte it points to
after boot -- \texttt{dsp0} -- will never be accessed: The DSP is decremented first with
two \texttt{dex} instructions, and then the new value is placed on the stack. This
means that the initial byte is garbage and can be considered part of the floodplain. 

\begin{lstlisting}[frame=single]
               +--------------+           
               |          ... |  
               +-            -+ 
               |              |   ...
               +-  (empty)   -+
               |              |  FE,X
               +-            -+ 
         ...   |              |  FF,X
               +==============+  
       \$0076  |           LSB|  00,X   <-- DSP (X Register)
               +-    TOS     -+ 
       \$0077  |           MSB|  01,X
               +==============+ 
       \$0078  |  (garbage)   |  02,X   <-- DSP0 
               +--------------+           
       \$0079  |              |  03,X
               + (floodplain) + 
       \$007A  |              |  04,X
               +--------------+           
\end{lstlisting}

\textit{Snapshot of the Data Stack with one entry as Top of the Stack (TOS). The
DSP has been increased by one and the value written.}

Note that the 65c02 system stack -- used as the Return Stack (RS) by Tali --
pushes the MSB on first and then the LSB (preserving little endian), so the_
basic structure is the same for both stacks. 

Because of this stack design, the second entry (`next on stack', NOS) starts at
\texttt{02,X} and the third entry (`third on stack', 3OS) at \texttt{04,X}. 

\subsection{Underflow detection} 
In contrast to Tali Forth 1, this version contains underflow detection for most
words. It does this by comparing the Data Stack Pointer (X) to values that it
must be smaller than (because the stack grows towards 0000). For instance, to
make sure we have one element on the stack, we write

\begin{lstlisting}[frame=single]
                cpx \#dsp0-1
                bmi okay

                lda \#11         ; error string for underflow
                jmp error
okay:
                (...)
\end{lstlisting}

For the most common cases, this gives us:

\begin{tabular}{ l c }
        1  cell   &  dsp0-1\\
        2  cells  &  dsp0-3\\
        3  cells  &  dsp0-5\\
\end{tabular}

Though underflow detection slows the code down slighly, it adds enormously to
the stability of the program.

\subsection{Double cell values}

The double cell is stored on top of the single cell.
Note this places the sign bit at the beginning of the byte below the DSP.

\begin{lstlisting}[frame=single]
               +---------------+
               |               |  
               +===============+  
               |            LSB|  \$0,x   <-- DSP (X Register) 
               +-+  Top Cell  -+         
               |S|          MSB|  \$1,x
               +-+-------------+ 
               |            LSB|  \$2,x
               +- Bottom Cell -+         
               |            MSB|  \$3,x   
               +===============+ 
\end{lstlisting}

Tali Forth 2 does not check for overflow, which in normal operation is too rare to
justify the computing expense. 

% ---------------------------------------
\section{Dictionary}


Tali Forth 2 follows the traditional model of a Forth dictionary -- a linked list
of words terminated with a zero pointer. The headers and code are kept separate
to allow various tricks in the code.


\subsection{Elements of the Header}

Each header is at least eight bytes long:

\begin{lstlisting}[frame=single]
              8 bit     8 bit
               LSB       MSB
 nt\_word ->  +--------+--------+
          +0 | Length | Status |
             +--------+--------+
          +2 | Next Header     | nt\_next\_word
             +-----------------+
          +4 | Start of Code   | xt\_word 
             +-----------------+
          +6 | End of Code     | z\_word
             +--------+--------+
          +8 | Name   |        |
             +--------+--------+
             |        |        |
             +--------+--------+
             |        |  ...   |
          +n +--------+--------+
\end{lstlisting}

Each word has a \texttt{name token} (nt, \texttt{nt\_word} in the code) that
points to the first byte of the header. This is the length of the word's name
string, which is limited to 255 characters. 

The second byte in the header (index 1) is the \texttt{status byte}. It is created by
the flags defined in the file \texttt{definitions.asm}: 

\begin{tabular}{ l c }
        CO & Compile Only\\
        IM & Immediate Word\\
        NN & Never Native Compile\\
        AN & Always Native Compile\\
\end{tabular}

Note there are currently four bits unused. The status byte is followed by the
\textbf{pointer to the next header} in the linked list, which makes it the named
token of the next word. A \texttt{0000} in this position signales the end of the
linked list, which by convention is the word \texttt{bye}. 

This is followed by the current word's \textbf{execution token} (xt,
\texttt{xt\_word}) that points to the start of the actual code. Some words that
have the same functionality point to the same code block. The \textbf{end of the
code} is referenced through the next pointer (\texttt{z\_word}) to enable native
compilation of the word if allowed. 

The \textbf{name string} starts at the eighth byte. The string is \textit{not}
zero-terminated. By default, the strings of Tali Forth 2 are lower case, but
case is respected for words the user defines, so `quarian' is a different words
than `QUARIAN'. 


\subsection{Structure of the Header List}

Tali Forth 2 distinguishes between three different list sources: The
\textbf{native words} that are hard-coded in the file
\texttt{native\_words.asm}, the \textbf{forth words} which are defined as
high-level words and then generated at run-time when Tali Forth starts up, and
\textbf{user words} in the file \texttt{user\_words.asm} which is empty when
Tali Forth ships. 

Tali has an unusually high number of native words in an attempt to make the
Forth as fast as possible on the 65c02. The first word in the list -- the one
that is checked first -- is always \texttt{drop}, the last one -- the one
checked for last -- is always \texttt{bye}. The words which are (or are assumed
to be) used more than others come first. Since humans are slow, words that are
used more interactively like \texttt{words} come later. 

The list of Forth words ends with the intro string. This functions as a
primitive form of a self-test: If you see the string and only the string, the
compilation of the Forth words worked.


% ---------------------------------------
\section{Memory Map}


Tali Forth 2 was developed with a simple 32 KiB RAM, 32 KiB ROM design. 

\begin{lstlisting}[frame=single]
    \$0000  +-------------------+  ram\_start, zpage, user0
           |  User varliables  |
           +-------------------+  
           |                   |
           |  \^  Data Stack    |  <-- dsp
           |  |                |
    \$0078  +-------------------+  dsp0, stack
           |                   |
           |   (Reserved for   |
           |      kernel)      |
           |                   |
    \$0100  +===================+  
           |                   |
           |  \^  Return Stack  |  <-- rsp 
           |  |                |
    \$0200  +-------------------+  rsp0, buffer, buffer0
           |  |                |
           |  v  Input Buffer  |
           |                   |
    \$0300  +-------------------+  cp0
           |  |                |
           |  v  Dictionary    |
           |       (RAM)       |
           |                   |
           ~~~~~~~~~~~~~~~~~~~~~  <-- cp
           |                   |
           |                   |
           |                   |
    \$7fff  #####################  ram\_end
    \$8000  |                   |  forth, code0
           |                   |
           |                   |
           |    Tali Forth     |
           |     (24 KiB)      |
           |                   |
           |                   |
    \$e000  +-------------------+
           |                   |  kernel\_putc, kernel\_getc   
           |      Kernel       |
           |                   |
    \$f000  +-------------------+  
           |   I/O addresses   |
           +-------------------+     
           |                   |
           |      Kernel       |
           |                   |
    \$fffa  +-------------------+     
           |  65c02 vectors    |
    \$ffff  +-------------------+     
\end{lstlisting}


% ---------------------------------------
\section{Input}

Tali Forth 2, like Liara Forth, follows the ANSI input model with
\texttt{refill} instead of older forms. There are up to four possible input
sources in Forth (see C\&D p. 155):

\begin{enumerate}
        \item The keyboard (`user input device')
        \item A character string in memory
        \item A block file
        \item A text file
\end{enumberate}

To check which one is being used, we first call \texttt{blk} which gives us the
number of a mass storage block being used, or 0 for the `user input device'
(keyboard). In the second case, we use SOURCE-ID to find out where input is
coming from: 0 for the keyboard, \texttt{-1 (\$FFFF)} for a string in memory,
and a number \texttt{n} for a file-id. Since Tali currently doesn't support
blocks, we can skip the \texttt{blk} instruction and go right to \texttt{source-id}. 


\subsection{Starting up}

The intial commands after reboot flow into each other: \texttt{cold} to
\texttt{abort} to \texttt{quit}. This is the same as with pre-ANSI Forths.
However, \texttt{quit} now calls \texttt{refill} to get the input.
\texttt{refill} does different things based on which of the four input sources
(see above) is active: 

\begin{description} 
        \item [Keyboard entry] This is the default. Get line of input via
                \texttt{accept} and return \texttt{true} even if the input string was
                empty.
        \item [\texttt{evaluate} string] Return a FALSE flag.
        \item [Input from a buffer] Not implemented at this time.
        \item [Input from a file] Not implemented at this time.
\end{description}


\subsection{The Command Line Interface}

Tali Forth accepts input lines of up to 256 characters. The address of the
current input buffer is stored in \texttt{cib} and is either \texttt{ibuffer1}
or \texttt{ibuffer2}, each of which is 256 bytes long. The length of the current
buffer is stored in \texttt{ciblen} -- this is the address that \texttt{>in}
returns. 

When a new line is entered, the address in \texttt{cib} is swapped, and the
contents of \texttt{ciblen} are moved to \texttt{`piblen} (for "previous input
buffer"). \texttt{`ciblen} is set to zero.  When the previous entry is
requested, the address in \texttt{cib} is swapped back, and \texttt{ciblen} and
\texttt{piblen} are swapped as well.  \texttt{SOURCE}by default returns
\texttt{cib} and \texttt{ciblen} as the address and length of the input buffer. 

(http://forth.sourceforge.net/standard/dpans/a0006.htm)
(http://forth.sourceforge.net/standard/dpans/dpansa6.htm#A.6.1.2216)

At some point, this system might be expanded to a real history list.


\subsection{\texttt{save-input} and \texttt{restore-input}}

(see http://forth.sourceforge.net/standard/dpans/dpansa6.htm#A.6.2.2182)



\subsection{evaluate}

(Automatically calls SAVE-INPUT and RESTORE-INPUT)
(http://forth.sourceforge.net/standard/dpans/a0006.htm)


\subsection{state}

(http://forth.sourceforge.net/standard/dpans/dpans6.htm#6.1.2250)



% ---------------------------------------
\section{Create/Does}

\texttt{create/does>} is the most complex, but also most powerful part of Forth.
Understanding how it works in Tali Forth is important if you want to be able to
modify the code. In this text, we walk through the generation process for a
Subroutine Threaded Code (STC) such as Tali Forth. For a more general take, see
Brad Rodriguez' series of articles at
\href{http://www.bradrodriguez.com/papers/moving3.htm}{http://www.bradrodriguez.com/papers/moving3.htm}.
There is a discussion of this walkthrough at
\href{http://forum.6502.org/viewtopic.php?f=9&t=3153}{http://forum.6502.org/viewtopic.php?f=9&t=3153}. 

We start with the following standard example, the Forth version of \texttt{CONSTANT: 

\begin{lstlisting}[frame=single]
        : constant create , does> @ ; 
\end{lstlisting}

We examine this in three phases or "sequences", based on Derick and Baker (see
Rodriguez for details):   


\subsubsection{SEQUENCE I: Compiling the word \texttt{CONSTANT}}

CONSTANT is a "defining word", one that makes new words. In pseudocode, and
ignoring any compilation to native 65c02 assembler, the above compiles to: 
\begin{lstlisting}[frame=single]
        ((Header "CONSTANT")) 
        jsr CREATE
        jsr COMMA
        jsr (DOES>)         ; from DOES>
   a:   jsr DODOES          ; from DOES>
   b:   jsr FETCH
        rts
\end{lstlisting}

To make things easier to explain later, we've added the labels `a' and `b' in
the listing. Note that \texttt{does>} is an immediate word that adds not one, but two
subroutine jumps, one to \texttt{(does>)} and one to \texttt{dodoes}, which is a pre-defined
system routine like \texttt{dovar}. we'll get to it later.

\footnote{In Tali Forth, a number of words such as \texttt{defer} are
`hand-compiled', that is, instead of using forth such (in this case, 

\begin{lstlisting}[frame=single]
        : defer create ['] abort , does> @ execute ; 
\end{lstlisting}

we write an opimized assembler version ourselves (see the actual \texttt{defer}
code). In these cases, we need to use \texttt{(does>)} and \texttt{dodoes}
instead of \texttt{does>} as well.}


\subsubsection{SEQUENCE II: Executing the word CONSTANT / creating LIFE }

Now when we execute

\begin{lstlisting}[frame=single]
        42 constant life
\end{lstlisting}

this pushes the RTS of the calling routine -- call it `main' -- to the 65c02's
stack (the Return Stack, as Forth calls it), which now looks like this:

\begin{lstlisting}[frame=single]
        ((1)) RTS               ; to main routine 
\end{lstlisting}

Without going into detail, the first two subroutine jumps of \texttt{constant} give us
this word: 

\begin{lstlisting}[frame=single]
        ((Header "LIFE"))
        jsr DOVAR               ; in CFA, from LIFE's CREATE
        4200                    ; in PFA (little-endian)
\end{lstlisting}

Next, we \texttt{jsr} to \texttt{(does>)}. The address that this pushes on the Return Stack is
the instruction of \texttt{constant} we had labeled `a'. 

\begin{lstlisting}[frame=single]
        ((2)) RTS to CONSTANT ("a") 
        ((1)) RTS to main routine 
\end{lstlisting}

Now the tricks start. \texttt{(does>)} takes this address off the stack and uses
it to replace the \texttt{dovar jsr} target in the CFA of our freshly created
\texttt{life} word. We now have this: 

\begin{lstlisting}[frame=single]
        ((Header "LIFE"))
        jsr a                   ; in CFA, modified by (DOES>)
   c:   4200                    ; in PFA (little-endian)
\end{lstlisting}

Note we added a label `c'. Now, when \texttt{(does>) reaches its own
\texttt{rts}, it finds the RTS to the main routine on its stack. This is Good
Thingi\textsuperscript{TM}, because it aborts the execution of the rest of
\texttt{constant}, and we don't want to do \texttt{dodoes} or \texttt{fetch}
now.  We're back at the main routine. 


\subsubsection{SEQUENCE III: Executing LIFE}

Now we execute the word \texttt{life} from our `main' program. In a STC Forth
such as Tali Forth, this executes a subroutine jump.

\begin{lstlisting}[frame=single]
        jsr LIFE
\end{lstlisting}

The first thing this call does is push the return address to the main routine
on the 65c02's stack: 

\begin{lstlisting}[frame=single]
        ((1)) RTS to main
\end{lstlisting}

The CFA of \texttt{life} executes a subroutine jump to label `a' in
\texttt{constant}. This pushes the \texttt{rts} of \texttt{life} on the 65c02's
stack:

\begin{lstlisting}[frame=single]
        ((2)) RTS to LIFE ("c")
        ((1)) RTS to main
\end{lstlisting}

This \texttt{jsr} to a lands us at the subroutine jump to \texttt{dodoes}, so
the return address to \texttt{constant} gets pushed on the stack as well. We had
given this instruction the label `b'. After all of this, we have three addresses
on the 65c02's stack: 

\begin{lstlisting}[frame=single]
        ((3)) RTS to CONSTANT ("b") 
        ((2)) RTS to LIFE ("c") 
        ((1)) RTS to main
\end{lstlisting}

\texttt{dodoes} pops address `b' off the 65c02's stack and puts it in a nice
safe place on Zero Page, which we'll call `z'. More on that in a moment. First,
\texttt{dodoes}.  pops the \texttt{RTS} to \texttt{life}. This is `c', the
address of the PFA or \texttt{life}, where we stored the payload of this
constant. Basically, \texttt{dodoes} performs a \texttt{dovar} here, and
pushes `c' on the Data Stack. Now all we have left on the 65c02's stack is the
\texttt{rts} to the main routine.  
 
\begin{lstlisting}[frame=single]
        [1] RTS to main
\end{lstlisting}

This is where `z' comes in, the location in Zero Page where we stored address
`b' of \texttt{constant}. Remember, this is where \texttt{constant}'s own PFA
begins, the \texttt{fetch} command we had originally codes after \texttt{does>}
in the very first definition. The really clever part: We perform an indirect
\texttt{jmp} -- not a \texttt{jsr}! -- to this address.

\begin{lstlisting}[frame=single]
        jmp (z) 
\end{lstlisting}

Now \texttt{constant}'s little payload programm is executed, the subroutine jump
to \texttt{fetch}. Since we just put the PFA (`c') on the Data Stack,
\texttt{fetch} replaces this by 42, which is what we were aiming for all along.
And since \texttt{constant} ends with a \texttt{rts}, we pull the last remaining
address off the 65c02's stack, which is the return address to the main routine
where we started. And that's all. 

Put together, this is what we have to code: 

\begin{description}

        \item [\texttt{does>}:] Compiles a subroutine jump to \texttt{(does>)},
                then compiles a subroutine jump to \texttt{dodoes}.

        \item [\texttt{(does>}):] Pops the stack (address of subroutine jump to
                \texttt{DODOES in \texttt{constant), increase this by one,
                replace the original \texttt{dovar jump target in \texttt{life}. 

        \item [\texttt{dodoes}:] Pop stack (\texttt{constant}'s PFA), increase
                address by one, store on Zero Page; pop stack (\texttt{life}'s
                PFA), increase by one, store on Data Stack; \texttt{jmp} to
                address we stored in Zero Page. 

\end{description}

Remember we have to increase the addresses by one because of the way
\texttt{jsr} stores the return address for \texttt{rts} on the stack on the
65c02: It points to the third byte of the \texttt{jsr} instruction itself, not
the actual return address.  This can be annoying, because it requires a sequence
like:

\begin{lstlisting}[frame=single]
        inc z
        bne +
        inc z+1 
*       (...) 
\end{lstlisting}

Note that with most words in Tali Forth, as any STC Forth, the distinction
between PFA and CFA is meaningless or at least blurred, because we go native
anyway. It is only with words generated by \texttt{create/does>} where this
really makes sense.

% ---------------------------------------
\section{Branches} 
\footnote{This section and the next one are based on a discussion at
\href{http://forum.6502.org/viewtopic.php?f=9&t=3176}
{http://forum.6502.org/viewtopic.php?f=9&t=3176}, see there for more details.
Another take on this subject that handles things a bit differently is at
\href{http://blogs.msdn.com/b/ashleyf/archive/2011/02/06/loopty-do-i-loop.aspx}{http://blogs.msdn.com/b/ashleyf/archive/2011/02/06/loopty-do-i-loop.aspx}
}

For \texttt{if/then}, we need to compile something called a `conditional forward
branch', traditionally called \texttt{0branch}. Then, at run-time, if the value
on the Data Stack is false (flag is zero), the branch is taken (`branch on
zero', therefore the name). Execpt that we don't have the target of that branch
yet -- it will later be added by \texttt{then}. For this to work, we remember
the address after the \texttt{0branch} instruction during the compilation of
\texttt{if}. this is put on the Data Stack, so that \texttt{then} knows where to
compile it's address in the second step.  Until then, a dummy value is compiled
after \texttt{0branch} to reserve the space we need. 

In Forth, this can be realized by

\begin{lstlisting}
        : if  postpone 0branch here 0 , ; immediate
\end{lstlisting}

and 

\begin{lstlisting}
        : then here swap ! ; immediate
\end{lstlisting}

Note \texttt{then} doesn't actually compile anything at the location in memory
where it is at. It's job is simply to help \texttt{if} out of the mess it
created.  If we have an \texttt{else}, we have to add an unconditional
\texttt{branch} and manipulate the address that \texttt{if} left on the Data
Stack. The Forth for this is: 

\begin{lstlisting}
        : else  postpone branch here 0 , here rot ! ; immediate
\end{lstlisting}

Note that \texttt{then} has no idea what has just happened, and just like before
compiles its address where the value on the top of the Data Stack told it to --
except that this value now comes from \texttt{else}, not \texttt{if}. 


\subsection{Loops} 

Loops are far more complicated, because we have \texttt{do} \texttt{?do}
\texttt{loop} \texttt{+loop}, \texttt{unloop}, and \texttt{leave} to take care
of. These can call up to three addresses: One for the normal looping action
(\texttt{loop}/\texttt{+loop}), one to skip over the loop at the beginning
(\texttt{?do}) and one to skip out of the loop (\texttt{leave}). 

Based on a suggestion by Garth Wilson, we begin each loop in run-time by saving
the address after the whole loop construct to the Return Stack. That way,
\texttt{leave} and \texttt{?do} know where to jump to when called, and we don't
interfere with any \texttt{if/then} structures. On top of that address, we place
the limit and start values for the loop. 

The key to staying sane while designing these constructs is to first make
a list of what we want to happen at compile-time and what at run-time. Let's
start with a simple \texttt{do}/\texttt{loop}.

\subsubsection{\texttt{do} at compile-time:}

\begin{itemize}

        \item Remember current address (in other words, \texttt{here}) on the
                Return Stack (!) so we can later compile the code for the
                post-loop address to the Return Stack

        \item Compile some dummy values to reserve the space for said code

        \item Compile the run-time code; we'll call that fragment (\texttt{do})

        \item Push the current address (the new \texttt{here}) to the Data Stack
                so \texttt{loop} knows where the loop contents begin

\end{itemize}

\subsubsection{\texttt{do} at run-time:}

\begin{itemize}
        \item - Take limit and start off Data Stack and push them to the Return Stack
\end{itemize}

Since \texttt{loop} is just a special case of \texttt{+loop} with an index of
one, we can get away with considering them at the same time. 


\subsubsection{\texttt{loop} at compile time:}

\begin{itemize}

        \item - Compile the run-time part \texttt{(+loop)}

        \item - Consume the address that is on top of the Data Stack as the jump
                target for normal looping and compile it

        \item Compile \texttt{unloop} for when we're done with the loop, getting
                rid of the limit/start and post-loop addresses on the Return
                Stack 

        \item  Get the address on the top of the Return Stack which points to
                the dummy code compiled by \texttt{do}

        \item At that address, compile the code that pushes the address after
                the list construct to the Return Stack at run-time

\end{itemize}

\subsubsection{\texttt{loop} at run-time (which is \texttt{(+loop)}) }

\begin{itemize}

        \item Add loop step to count

        \item Loop again if we haven't crossed the limit, otherwise continue
                after loop

\end{itemize}

At one glance, we can see that the complicated stuff happens at compile-time.
This is good, because we only have to do that once for each loop. 

In Tali Forth, these routines are coded in assembler. With this setup,
\texttt{unloop} becomes simple (six PLAs -- four for the limit/count of
\texttt{do}, two for the address pushed to the stack just before it) and
\texttt{leave} even simpler (four PLAs for the} address). 

