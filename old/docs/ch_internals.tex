% Internals of Tali Forth 2
% Scot W. Stevenson 
% This version: 13. June 2018
% ---------------------------------------
\section{Stack}\index{stack|textbf}

Tali Forth 2 uses the lowest part of the top half of Zero Page\index{Zero Page}
for the Data Stack (DS). This leaves the lower half of the Zero Page for any
kernel stuff the user might require. The DS therefore grows towards the initial
user variables. See the file \texttt{definitions.asm} for details.  Because of
the danger of underflow\index{underflow}, it is recommended that the user
kernel's variables are keep closer to \texttt{\$0100} than to \texttt{\$007f}.

The X register\index{X register} is used as the Data Stack Pointer (DSP). It
points to the least significant byte of the current top element of the stack
(`Top of the Stack', TOS).\footnote{In the first versions of Tali\index{Tali
Forth 1}, the DSP pointed to the next \textit{free} element of the stack. The
new system makes detecting underflow easier and parallels the structure of Liara
Forth\index{Liara Forth}.}

Initially, the DSP points to \texttt{\$78}, not \texttt{\$7F} as might be
expected. This provides a few bytes as a `floodplain'\index{floodplain (stack)}
in case of underflow\index{underflow}. The initial value of the DSP is defined
as \texttt{dsp0} in the code.

\subsection{Single cell values} Since the cell size is 16 bits, each stack entry
consists of two bytes. They are stored little endian\index{little endian} (least
significant byte first). Therefore, the DSP points to the LSB of the current
TOS.\footnote{Try reading that last sentence to a friend who isn't into
computers. Aren't abbreviations fun?}

Because the DSP points to the current top of the stack, the byte it points to
after boot -- \texttt{dsp0} -- will never be accessed: The DSP is decremented first with
two \texttt{dex} instructions, and then the new value is placed on the stack. This
means that the initial byte is garbage and can be considered part of the floodplain. 

\begin{lstlisting}[frame=single]
              +--------------+           
              |          ... |  
              +-            -+ 
              |              |   ...
              +-  (empty)   -+
              |              |  FE,X
              +-            -+ 
        ...   |              |  FF,X
              +==============+  
       $0076  |           LSB|  00,X   <-- DSP (X Register)
              +-    TOS     -+ 
       $0077  |           MSB|  01,X
              +==============+ 
       $0078  |  (garbage)   |  02,X   <-- DSP0 
              +--------------+           
       $0079  |              |  03,X
              + (floodplain) + 
       $007A  |              |  04,X
              +--------------+           
\end{lstlisting}

\textit{Snapshot of the Data Stack with one entry as Top of the Stack (TOS). The
DSP has been increased by one and the value written.}

Note that the 65c02 system stack -- used as the Return Stack (RS) by Tali --
pushes the MSB on first and then the LSB (preserving little endian), so the
basic structure is the same for both stacks. 

Because of this stack design, the second entry (`next on stack', NOS) starts at
\texttt{02,X} and the third entry (`third on stack', 3OS) at \texttt{04,X}. 

\subsection{Underflow detection}\index{underflow!detection} In contrast to Tali
Forth 1\index{Tali Forth 1}, this version contains underflow detection for most
words. It does this by comparing the Data Stack Pointer (X)\index{X register} to
values that it must be smaller than (because the stack grows towards 0000). For
instance, to make sure we have one element on the stack, we write

\begin{lstlisting}[frame=lines]
                cpx #dsp0-1
                bmi okay
                jmp underflow
okay:
                (...)
\end{lstlisting}

\noindent For the most common cases, this gives us:

\begin{table}[h!]
\centering
\begin{tabular}{ | c | c | }
        \hline
        Test for &  Pointer offset\\
        \hline
        1 cell   &  dsp0-1\\
        2 cells  &  dsp0-3\\
        3 cells  &  dsp0-5\\
        4 cells  &  dsp0-7\\
        \hline
\end{tabular}
        \caption{DSP values for underflow testing}
        \label{table_uf}
\end{table}

Underflow detection\index{underflow!detection} adds seven bytes 
to the words that have it. However, it increases the stability of the
program enormously.

\subsection{Double cell values}\index{double cell}

The double cell is stored on top of the single cell.  Note this places the sign
bit at the beginning of the byte below the DSP.

\begin{lstlisting}[frame=single]
              +---------------+
              |               |  
              +===============+  
              |            LSB|  $0,x   <-- DSP (X Register) 
              +-+  Top Cell  -+         
              |S|          MSB|  $1,x
              +-+-------------+ 
              |            LSB|  $2,x
              +- Bottom Cell -+         
              |            MSB|  $3,x   
              +===============+ 
\end{lstlisting}

Tali Forth 2 does not check for overflow\index{overflow}, which in normal
operation is too rare to justify the computing expense. 

% ---------------------------------------
\section{Dictionary}\index{dictionary|textbf}

Tali Forth 2 follows the traditional model of a Forth dictionary -- a linked list
of words terminated with a zero pointer. The headers and code are kept separate
to allow various tricks in the code.


\subsection{Elements of the Header}\index{header|textbf}

Each header is at least eight bytes long:

\begin{lstlisting}[frame=single]
              8 bit     8 bit
               LSB       MSB
nt_word ->  +--------+--------+
         +0 | Length | Status |
            +--------+--------+
         +2 | Next Header     | nt_next_word
            +-----------------+
         +4 | Start of Code   | xt_word 
            +-----------------+
         +6 | End of Code     | z_word
            +--------+--------+
         +8 | Name   |        |
            +--------+--------+
            |        |        |
            +--------+--------+
            |        |  ...   |
         +n +--------+--------+
\end{lstlisting}

Each word has a \texttt{name token}\index{name token} (nt\index{nt|see {name
token}}, \texttt{nt\_word} in the code) that
points to the first byte of the header. This is the length of the word's name
string, which is limited to 255 characters. 

The second byte in the header (index 1) is the \texttt{status byte}. It is created by
the flags\index{header!flags} defined in the file \texttt{definitions.asm}: 

\begin{table}[h!]
        \label{tab:table_flags}
        \centering
\begin{tabular}{ | c | l | }
        \hline
        Flag & Function\\
        \hline
        CO & Compile Only\\
        IM & Immediate Word\\
        NN & Never Native Compile\\
        AN & Always Native Compile\\
        UF & Underflow dectection\\
        \hline
\end{tabular}
        \caption{Header flags}
\end{table}

Note there are currently three bits unused. The status byte\index{status byte}
is followed by the \textbf{pointer to the next header} in the linked list, which
makes it the name token\index{name token} of the next word. A \texttt{0000} in
this position signales the end of the linked list, which by convention is the
word \texttt{bye}. 

This is followed by the current word's \textbf{execution token}\index{execution
token} (xt\index{xt|see {execution token}},
\texttt{xt\_word}) that points to the start of the actual code. Some words that
have the same functionality point to the same code block. The \textbf{end of the
code} is referenced through the next pointer (\texttt{z\_word}) to enable native
compilation of the word if allowed. 

The \textbf{name string} starts at the eighth byte. The string is \textit{not}
zero-terminated. By default, the strings of Tali Forth 2 are lower
case\index{case, lower}, but case is respected for words the user defines, so
`quarian' is a different words than `QUARIAN'. 


\subsection{Structure of the Header List}

Tali Forth 2 distinguishes between three different list sources: The
\textbf{native words}\index{native words} that are hard-coded in the file
\texttt{native\_words.asm}, the \textbf{Forth words}\index{Forth words} which
are defined as high-level words and then generated at run-time when Tali Forth
starts up, and \textbf{user words} in the file
\texttt{user\_words.asm}\index{user words}. 

Tali has an unusually high number of native words in an attempt to make the
Forth as fast as possible on the 65c02\index{65c02}. The first word in the list
-- the one that is checked first -- is always
\texttt{drop}\index{drop@\texttt{drop}}, the last one -- the one checked for
last -- is always \texttt{bye}\index{bye@\texttt{bye}}. The words which are (or
are assumed to be) used more than others come first. Since humans are slow,
words that are used more interactively like
\texttt{words}\index{words@\texttt{words}} come later. 

The list of Forth words ends with the intro strings\index{intro string}. This
functions as a primitive form of a self-test\index{self-test}: If you see the
string and only the string, the compilation of the Forth words worked.

\pagebreak

% ---------------------------------------
\section{Memory Map}\index{memory map|textbf}

Tali Forth 2 was developed with a simple 32 KiB RAM\index{RAM}, 32 KiB
ROM\index{ROM} design. 

\begin{lstlisting}[frame=single]
   $0000  +-------------------+  ram_start, zpage, user0
          |  User varliables  |
          +-------------------+  
          |                   |
          |  ^  Data Stack    |  <-- dsp
          |  |                |
   $0078  +-------------------+  dsp0, stack
          |                   |
          |   (Reserved for   |
          |      kernel)      |
          |                   |
   $0100  +===================+  
          |                   |
          |  ^  Return Stack  |  <-- rsp 
          |  |                |
   $0200  +-------------------+  rsp0, buffer, buffer0
          |  |                |
          |  v  Input Buffer  |
          |                   |
   $0300  +-------------------+  cp0
          |  |                |
          |  v  Dictionary    |
          |       (RAM)       |
          |                   |
          ~~~~~~~~~~~~~~~~~~~~~  <-- cp
          |                   |
          |                   |
          +-------------------+     
          |                   |
          | ACCEPT history    | 
          |                   |
   $7FFF  #####################  ram_end
   $8000  |                   |  forth, code0
          |                   |
          |                   |
          |    Tali Forth     |
          |     (24 KiB)      |
          |                   |
          |                   |
   $E000  +-------------------+
          |                   |  kernel_putc, kernel_getc   
          |      Kernel       |
          |                   |
   $F000  +-------------------+  
          |   I/O addresses   |
          +-------------------+     
          |                   |
          |      Kernel       |
          |                   |
   $FFFA  +-------------------+     
          |  65c02 vectors    |
   $FFFF  +-------------------+     
\end{lstlisting}

Note that some of these values are hard-coded into the test\index{testing}
suite; see the file \texttt{definitions.txt} for details.

\pagebreak

% ---------------------------------------
\section{Input}

Tali Forth 2, like Liara Forth, follows the ANSI\index{ANSI Forth} input model
with \texttt{refill} instead of older forms. There are up to four possible input
sources in Forth (see C\&D p. 155):

\begin{enumerate}
        \item The keyboard (`user input device')\index{keyboard}
        \item A character string in memory
        \item A block file\index{file!block}
        \item A text file\index{file!text}
\end{enumerate}

To check which one is being used, we first call
\texttt{blk}\index{blk@\texttt{blk}} which gives us the number of a mass storage
block being used, or 0 for the `user input device' (keyboard). In the second
case, we use SOURCE-ID to find out where input is coming from: 0 for the
keyboard, \texttt{-1 (\$FFFF)} for a string in memory, and a number \texttt{n}
for a file-id\index{file-id}. Since Tali currently doesn't support blocks, we
can skip the \texttt{blk} instruction and go right to \texttt{source-id}. 

One gotcha with Tali Forth's input is that current it only sees spaces, but not
other whitespace, as delimiters. This means that Forth text files that are fed
to Tali should not contain tabs. This behavior might be changed in the future.


\subsection{Starting up}\index{booting}

The intial commands after reboot flow into each other: \texttt{cold} to
\texttt{abort} to \texttt{quit}. This is the same as with pre-ANSI Forths.
However, \texttt{quit} now calls \texttt{refill} to get the input.
\texttt{refill} does different things based on which of the four input sources
(see above) is active: 

\begin{description} 
        \item [Keyboard entry] This is the default. Get line of input via
                \texttt{accept} and return \texttt{true} even if the input string was
                empty.
        \item [\texttt{evaluate} string] Return a \texttt{false} flag.
        \item [Input from a buffer] Not implemented at this time.
        \item [Input from a file] Not implemented at this time.
\end{description}


\subsection{The Command Line Interface}\index{command line interface}

Tali Forth accepts input\index{input|textbf} lines of up to 256 characters. The
address of the current input buffer\index{current input buffer} is stored in
\texttt{cib}\index{cib|see {current input buffer}}.  The length of the current
buffer is stored in \texttt{ciblen} -- this is the address that
\texttt{>in}\index{>in@\texttt{>in}} returns.
\texttt{source}\index{source@\texttt{source}} by default returns \texttt{cib}
and \texttt{ciblen} as the address and length of the input buffer. 


\subsection{\texttt{evaluate}}\index{evaluate@\texttt{evaluate}}

\texttt{evaluate} is used to execute commands that are in a string. A simple
example would be:

\begin{lstlisting}[frame=lines]
        s" 1 2 + ." evaluate
\end{lstlisting}

\noindent Tali Forth uses \texttt{evaluate} to load high-level Forth words from
the file
\texttt{forth\_words.asc}\index{forth\_words.asc@\texttt{forth\_words.asc}} and
extra, user-defined words from
\texttt{user\_words.asc}\index{user\_words.asc@\texttt{user\_words.asc}}. 

% ---------------------------------------
\section{\texttt{create}/\texttt{does>}}
\index{create@\texttt{create}|textbf}
\index{does>@\texttt{does>}|textbf}

\texttt{create/does>} is the most complex, but also most powerful part of Forth.
Understanding how it works in Tali Forth is important if you want to be able to
modify the code. In this text, we walk through the generation process for a
subroutine threaded code (STC)\index{threading} such as Tali Forth. For a more
general take, see Brad Rodriguez' series of articles at
\href{http://www.bradrodriguez.com/papers/moving3.htm}{http://www.bradrodriguez.com/papers/moving3.htm}.
There is a discussion of this walkthrough at
\href{http://forum.6502.org/viewtopic.php?f=9\&t=3153}{http://forum.6502.org/viewtopic.php?f=9\&t=3153}. 

We start with the following standard example, the Forth version of
\texttt{constant}:\index{constant@\texttt{constant}}. 

\begin{lstlisting}[frame=lines]
        : constant create , does> @ ; 
\end{lstlisting}

\noindent We examine this in three phases or "sequences", following Rodriguez
based on\cite{derickbaker}:

\subsubsection{Sequence 1: Compiling the word \texttt{constant}}

\texttt{constant} is a `defining word', one that makes new words. In pseudocode, and
ignoring any compilation to native 65c02 assembler, the above compiles to: 
\begin{lstlisting}[frame=lines]
        jsr CREATE
        jsr COMMA
        jsr (DOES>)         ; from DOES>
   a:   jsr DODOES          ; from DOES>
   b:   jsr FETCH
        rts
\end{lstlisting}

\noindent To make things easier to explain later, we've added the labels `a' and
`b' in the listing.\footnote{This example uses the word \texttt{(does>)}, which
in Tali Forth 2 is actually an internal routine that does not appear as a
separate word. This version is easier to explain.} \texttt{does>} is an
immediate word that adds not one, but two subroutine jumps, one to
\texttt{(does>)} and one to \texttt{dodoes}, which is a pre-defined system
routine like \texttt{dovar}. We'll discuss those later.

In Tali Forth, a number of words such as \texttt{defer} are
`hand-compiled', that is, instead of using forth such as

\begin{lstlisting}[frame=lines]
        : defer create ['] abort , does> @ execute ; 
\end{lstlisting}

\noindent we write an opimized assembler version ourselves (see the actual \texttt{defer}
code). In these cases, we need to use \texttt{(does>)} and \texttt{dodoes}
instead of \texttt{does>} as well.


\subsubsection{Sequence 2: Executing the word \texttt{constant}/creating
\texttt{life}}

Now when we execute

\begin{lstlisting}[frame=lines]
        42 constant life
\end{lstlisting}

\noindent this pushes the \texttt{rts} of the calling routine -- call it `main'
-- to the 65c02's stack (the Return Stack, as Forth calls it), which now looks
like this:

\begin{lstlisting}[frame=lines]
        (1) RTS               ; to main routine 
\end{lstlisting}

\noindent Without going into detail, the first two subroutine jumps of \texttt{constant} give us
this word: 

\begin{lstlisting}[frame=lines]
        (Header "LIFE")
        jsr DOVAR               ; in CFA, from LIFE's CREATE
        4200                    ; in PFA (little-endian)
\end{lstlisting}

\noindent Next, we \texttt{jsr} to \texttt{(does>)}. The address that this
pushes on the Return Stack is the instruction of \texttt{constant} we had
labeled `a'. 

\begin{lstlisting}[frame=lines]
        (2) RTS to CONSTANT ("a") 
        (1) RTS to main routine 
\end{lstlisting}

\noindent Now the tricks start. \texttt{(does>)} takes this address off the stack and uses
it to replace the \texttt{dovar jsr} target in the CFA of our freshly created
\texttt{life} word. We now have this: 

\begin{lstlisting}[frame=lines]
        (Header "LIFE")
        jsr a                   ; in CFA, modified by (DOES>)
   c:   4200                    ; in PFA (little-endian)
\end{lstlisting}

\noindent Note we added a label `c'. Now, when \texttt{(does>)} reaches its own
\texttt{rts}, it finds the \texttt{rtrs} to the main routine on its stack. This
is Good Thingi\textsuperscript{TM}, because it aborts the execution of the rest
of \texttt{constant}, and we don't want to do \texttt{dodoes} or \texttt{fetch}
now.  We're back at the main routine. 


\subsubsection{Sequence 3: Executing \texttt{life}}

Now we execute the word \texttt{life} from our `main' program. In a STC Forth
such as Tali Forth, this executes a subroutine jump.

\begin{lstlisting}[frame=lines]
        jsr LIFE
\end{lstlisting}

\noindent The first thing this call does is push the return address to the main routine
on the 65c02's stack: 

\begin{lstlisting}[frame=lines]
        (1) RTS to main
\end{lstlisting}

\noindent The CFA of \texttt{life} executes a subroutine jump to label `a' in
\texttt{constant}. This pushes the \texttt{rts} of \texttt{life} on the 65c02's
stack:

\begin{lstlisting}[frame=lines]
        (2) RTS to LIFE ("c")
        (1) RTS to main
\end{lstlisting}

\noindent This \texttt{jsr} to a lands us at the subroutine jump to \texttt{dodoes}, so
the return address to \texttt{constant} gets pushed on the stack as well. We had
given this instruction the label `b'. After all of this, we have three addresses
on the 65c02's stack: 

\begin{lstlisting}[frame=lines]
        (3) RTS to CONSTANT ("b") 
        (2) RTS to LIFE ("c") 
        (1) RTS to main
\end{lstlisting}

\noindent \texttt{dodoes} pops address `b' off the 65c02's stack and puts it in a nice
safe place on Zero Page, which we'll call `z'. More on that in a moment. First,
\texttt{dodoes} pops the \texttt{rts} to \texttt{life}. This is `c', the
address of the PFA or \texttt{life}, where we stored the payload of this
constant. Basically, \texttt{dodoes} performs a \texttt{dovar} here, and
pushes `c' on the Data Stack. Now all we have left on the 65c02's stack is the
\texttt{rts} to the main routine.  
 
\begin{lstlisting}[frame=lines]
        [1] RTS to main
\end{lstlisting}

\noindent This is where `z' comes in, the location in Zero Page where we stored address
`b' of \texttt{constant}. Remember, this is where \texttt{constant}'s own PFA
begins, the \texttt{fetch} command we had originally codes after \texttt{does>}
in the very first definition. The really clever part: We perform an indirect
\texttt{jmp} -- not a \texttt{jsr}! -- to this address.

\begin{lstlisting}[frame=lines]
        jmp (z) 
\end{lstlisting}

\noindent Now \texttt{constant}'s little payload programm is executed, the subroutine jump
to \texttt{fetch}. Since we just put the PFA (`c') on the Data Stack,
\texttt{fetch} replaces this by 42, which is what we were aiming for all along.
And since \texttt{constant} ends with a \texttt{rts}, we pull the last remaining
address off the 65c02's stack, which is the return address to the main routine
where we started. And that's all. 

Put together, this is what we have to code: 

\begin{description}
        \item [\texttt{does>}:] Compiles a subroutine jump to \texttt{(does>)},
                then compiles a subroutine jump to \texttt{dodoes}.

        \item [\texttt{(does>)}:] Pops the stack (address of subroutine jump to
                \texttt{dodoes} in \texttt{constant}, increase this by one,
                replace the original \texttt{dovar} jump target in \texttt{life}. 

        \item [\texttt{dodoes}:] Pop stack (\texttt{constant}'s PFA), increase
                address by one, store on Zero Page; pop stack (\texttt{life}'s
                PFA), increase by one, store on Data Stack; \texttt{jmp} to
                address we stored in Zero Page.\index{Zero Page} 
\end{description}

Remember we have to increase the addresses by one because of the way
\texttt{jsr} stores the return address for \texttt{rts} on the stack on the
65c02: It points to the third byte of the \texttt{jsr} instruction itself, not
the actual return address. This can be annoying, because it requires a sequence
like:

\begin{lstlisting}[frame=lines]
        inc z
        bne +
        inc z+1 
*       (...) 
\end{lstlisting}

\noindent Note that with most words in Tali Forth, as any STC Forth, the distinction
between PFA and CFA is meaningless or at least blurred, because we go native
anyway. It is only with words generated by \texttt{create/does>} where this
really makes sense.

% ---------------------------------------
\section{Control Flow} 

\subsection{Branches} 

For \texttt{if/then}, we need to compile something called a `conditional forward
branch', traditionally called \texttt{0branch}.\footnote{Many Forths now use the
words \texttt{cs-pick} and \texttt{cs-roll} instead of the \texttt{branch}
variants, see
\href{http://lars.nocrew.org/forth2012/rationale.html\#rat:tools:CS-PICK}
{http://lars.nocrew.org/forth2012/rationale.html\#rat:tools:CS-PICK}. Tali Forth
might switch to this construction in the future.} Then, at run-time, if the
value on the Data Stack is false (flag is zero), the branch is taken (`branch on
zero', therefore the name). Execpt that we don't have the target of that branch
yet -- it will later be added by \texttt{then}. For this to work, we remember
the address after the \texttt{0branch} instruction during the compilation of
\texttt{if}. This is put on the Data Stack, so that \texttt{then} knows where to
compile it's address in the second step. Until then, a dummy value is compiled
after \texttt{0branch} to reserve the space we need.\footnote{This section and
the next one are based on a discussion at
\href{http://forum.6502.org/viewtopic.php?f=9\&t=3176}
{http://forum.6502.org/viewtopic.php?f=9\&t=3176}, see there for more details.
Another take on this subject that handles things a bit differently is at
\href{http://blogs.msdn.com/b/ashleyf/archive/2011/02/06/loopty-do-i-loop.aspx}{http://blogs.msdn.com/b/ashleyf/archive/2011/02/06/loopty-do-i-loop.aspx}
}

In Forth, this can be realized by

\begin{lstlisting}[frame=lines]
        : if  postpone 0branch here 0 , ; immediate
\end{lstlisting}

\noindent and 

\begin{lstlisting}[frame=lines]
        : then here swap ! ; immediate
\end{lstlisting}

\noindent Note \texttt{then} doesn't actually compile anything at the location in memory
where it is at. It's job is simply to help \texttt{if} out of the mess it
created.  If we have an \texttt{else}, we have to add an unconditional
\texttt{branch} and manipulate the address that \texttt{if} left on the Data
Stack. The Forth for this is: 

\begin{lstlisting}[frame=lines]
        : else  postpone branch here 0 , here rot ! ; immediate
\end{lstlisting}

\noindent Note that \texttt{then} has no idea what has just happened, and just like before
compiles its address where the value on the top of the Data Stack told it to --
except that this value now comes from \texttt{else}, not \texttt{if}. 


\subsection{Loops} 

Loops are far more complicated, because we have \texttt{do}, \texttt{?do},
\texttt{loop}, \texttt{+loop}, \texttt{unloop}, and \texttt{leave} to take care
of. These can call up to three addresses: One for the normal looping action
(\texttt{loop}/\texttt{+loop}), one to skip over the loop at the beginning
(\texttt{?do}) and one to skip out of the loop (\texttt{leave}). 

Based on a suggestion by Garth Wilson, we begin each loop in run-time by saving
the address after the whole loop construct to the Return Stack. That way,
\texttt{leave} and \texttt{?do} know where to jump to when called, and we don't
interfere with any \texttt{if/then} structures. On top of that address, we place
the limit and start values for the loop. 

The key to staying sane while designing these constructs is to first make
a list of what we want to happen at compile-time and what at run-time. Let's
start with a simple \texttt{do}/\texttt{loop}.

\subsubsection{\texttt{do} at compile-time:}

\begin{itemize}

        \item Remember current address (in other words, \texttt{here}) on the
                Return Stack (!) so we can later compile the code for the
                post-loop address to the Return Stack

        \item Compile some dummy values to reserve the space for said code

        \item Compile the run-time code; we'll call that fragment (\texttt{do})

        \item Push the current address (the new \texttt{here}) to the Data Stack
                so \texttt{loop} knows where the loop contents begin

\end{itemize}

\subsubsection{\texttt{do} at run-time:}

\begin{itemize}
        \item Take limit and start off Data Stack and push them to the Return Stack
\end{itemize}

\noindent Since \texttt{loop} is just a special case of \texttt{+loop} with an index of
one, we can get away with considering them at the same time. 


\subsubsection{\texttt{loop} at compile time:}

\begin{itemize}

        \item Compile the run-time part \texttt{(+loop)}

        \item Consume the address that is on top of the Data Stack as the jump
                target for normal looping and compile it

        \item Compile \texttt{unloop} for when we're done with the loop, getting
                rid of the limit/start and post-loop addresses on the Return
                Stack 

        \item  Get the address on the top of the Return Stack which points to
                the dummy code compiled by \texttt{do}

        \item At that address, compile the code that pushes the address after
                the list construct to the Return Stack at run-time

\end{itemize}

\subsubsection{\texttt{loop} at run-time (which is \texttt{(+loop)}) }

\begin{itemize}

        \item Add loop step to count

        \item Loop again if we haven't crossed the limit, otherwise continue
                after loop

\end{itemize}

At one glance, we can see that the complicated stuff happens at compile-time.
This is good, because we only have to do that once for each loop. 

In Tali Forth, these routines are coded in assembler. With this setup,
\texttt{unloop} becomes simple (six \texttt{pla} instructions -- four for the
limit/count of \texttt{do}, two for the address pushed to the stack just before
it) and \texttt{leave} even simpler (four \texttt{pla} instructions for the
address). 

\section{Native Compiling}\index{compiling, native|textbf}

In a pure subroutine threaded code, higher-level words are merely a series of
subroutine jumps. For instance, the Forth word
\texttt{[char]}\index{[char]@\texttt{[char]}}, formally defined in high-level
Forth as

\begin{lstlisting}[frame=lines]
        : [char] char postpone literal ; immediate
\end{lstlisting}

\noindent in assembler is simply

\begin{lstlisting}[frame=lines]
                jsr xt_char
                jsr xt_literal
\end{lstlisting}

\noindent as an immediate, compile-only word. Theare are two obvious problems
with this method: First, it is slow, because each \texttt{jsr/rts} pair consumes
four bytes and 12 cycles overhead. Second, for smaller words, the jumps use far
more bytes than the actual code. Take for instance
\texttt{drop}\index{drop@\texttt{drop}}, which in its naive form is simply

\begin{lstlisting}[frame=lines]
                inx
                inx
\end{lstlisting}

\noindent for two bytes and four cycles. If we jump to this word as is assumed
with pure subroutine threaded Forth, we add four bytes and 12 cycles -- double
the space and three times the time required by the actual working code. (In
practice, it's even worse, because \texttt{drop} checks for
underflow\index{underflow}. The actual assembler code is

\begin{lstlisting}[frame=lines]
                cpx #dsp0-1
                bmi +
                lda #11         ; error code for underflow
                jmp error
*
                inx
                inx
\end{lstlisting}

\noindent for eleven bytes. We'll discuss the underflow checks further below.)

To get rid of this problem, Tali Forth supports \textbf{native
compiling}\index{compiling, native}. The system variable
\texttt{nc-limit}\index{nc-limit@\texttt{nc-limit}} sets
the threshhold up to which a word will be included not as a subroutine jump, but
machine language. Let's start with an example where \texttt{nc-limit} is set to
zero, that is, all words are compiled as subroutine jumps. Take a simple word
such as

\begin{lstlisting}[frame=lines]
        : aaa 0 drop ;
\end{lstlisting}

\noindent and check the actual code with \texttt{see}\index{see@\texttt{see}}

\begin{lstlisting}[frame=lines]
        see aaa
          nt: 7CD  xt: 7D8
         size (decimal): 6

        07D8  20 52 99 20 6B 88  ok
\end{lstlisting}

\noindent (The actual addresses might be different, this is from the ALPHA release).  Our
word \texttt{aaa} consists of two subroutine jumps, one to zero and one to
\texttt{drop}. Now, if we increase the threshhold to 20, we get different code,
as this console session shows:

\begin{lstlisting}[frame=lines]
        20 nc-limit !  ok
        : bbb 0 drop ;  ok
        see bbb
          nt: 7DF  xt: 7EA
         size (decimal): 17

        07EA  CA CA 74 00 74 01 E0 77  30 05 A9 0B 4C C7 AC E8
        07FA  E8  ok
\end{lstlisting}

\noindent Even though the definition of \texttt{bbb} is the same as \texttt{aaa}, we have
totally different code: The number 0001 is pushed to the Data Stack (the first
six bytes), then we check for underflow\index{underflow} (the next nine bytes),
and finally we \texttt{drop} by moving X\index{X register}, the Data Stack
Pointer. Our word is definitely longer, but have just saved 12 cycles.

To experiment with various parameters for native compiling, the Forth word
\texttt{words\&sizes}\index{words\&sizes@\texttt{words\&sizes}} is included in
\texttt{user\_words.fs} (but commented out by default). The Forth is:

\begin{lstlisting}[frame=lines]
: words&sizes ( -- )
        latestnt
        begin
                dup
        0<> while
                dup name>string type space
                dup wordsize u. cr
                2 + @
        repeat
        drop ;
\end{lstlisting}

\noindent An alternative is \texttt{see}\index{see@\texttt{see}}, which also
displays the length of a word. One way or another, changing \texttt{nc-limit}
should show differences in the Forth words.

\subsection{Return Stack special cases}\index{Return Stack}

There are a few words that cause problems with subroutine threaded
code\index{threading}: Those that access the Return Stack such as
\texttt{r>}\index{r>@\texttt{r>}}, \texttt{>r}\index{>r@\texttt{>r}},
\texttt{r@}\index{r\@@\texttt{r@}}, \texttt{2r>}\index{2r>@\texttt{2r>}}, and
\texttt{2>r}\index{2>r@\texttt{2>r}}. For them to work correctly, we first
have to remove the return address on the top of the stack, only to replace it
again before we return to the caller. This mechanism would normally prevent the
word from being natively compiled\index{compiling!native} at all, because we'd
try to remove a return address that doesn't exit.

This becomes clearer when we examine the code for \texttt{>r} (comments
removed):

\begin{lstlisting}[frame=lines]
xt_r_from:
                pla
                sta tmptos
                ply

                ; --- CUT FOR NATIVE CODING ---

                dex
                dex
                pla
                sta 0,x
                pla
                sta 1,x

                ; --- CUT FOR NATIVE CODING ---
     
                phy
                lda tmptos
                pha

z_r_from:       rts
\end{lstlisting}

\noindent The first three and last three instructions are purely for
housekeeping with subroutine threaded code. To enable this routine to be
included as native code, they are removed when native compiling is enabled by
the word \texttt{compile,}\index{compile,@\texttt{compile,}}. This leaves us
with just the six actual instructions in the center of the routine to be
compiled into the new word.

\subsection{Underflow stripping}\index{underflow!stripping}

As described above, every underflow check adds seven bytes to the
word being coded. Stripping this check by setting the
\texttt{uf-strip}\index{uf-strip@\texttt{uf-strip}} system variable to
\texttt{true}\index{true@\texttt{true}} simply removes these seven bytes from new
natively compiled words. 

It is possible, of course, to have lice and fleas at the some time. For
instance, this is the code for \texttt{>r}\index{>r@\texttt{>r}}:

\begin{lstlisting}[frame=lines]
xt_to_r:
                pla
                sta tmptos
                ply

                ; --- CUT HERE FOR NATIVE CODING ---

                cpx #dsp0-1
                bmi +
                jmp underflow
*
                lda 1,x
                pha
                lda 0,x
                pha

                inx
                inx

                ; --- CUT HERE FOR NATIVE CODING ---

                phy
                lda tmptos
                pha

z_to_r:         rts
\end{lstlisting}

\noindent This word has \textit{both} native compile stripping and underflow
detection. However, both can be removed from newly native code words, leaving
only the eight byte core of the word to be compiled.

% ---------------------------------------
\section{\texttt{cmove, cmove>} and \texttt{move}}

The three moving words \texttt{cmove}\index{cmove@\texttt{cmove}},
\texttt{cmove>}\index{cmove>@\texttt{cmove>}}, and
\texttt{move}\index{move@\texttt{move}} show subtle differences that can trip up
new users and are reflected by different code under the hood. \texttt{cmove} and
\texttt{cmove>} are the traditional Forth words that work on characters (which
in the case of Tali Forth are bytes), whereas \texttt{move} is a more modern
word that works on address units (which in our case is also bytes). 

If the source and destination regions show no overlap, all three words work the
same. However, if there is overlap, \texttt{cmove} and \texttt{cmove>}
demonstrate a behavior called ``propagation''\index{propagation} or
``clobbering''\index{clobbering}: Some of the characters are overwritten.
\texttt{move} does not show this behavior. This example shows the difference:

\begin{lstlisting}[frame=lines]
create testbuf char a c, char b c, char c c, char d c,  ( ok ) 
testbuf 4 type ( abcd ok ) 
testbuf dup char+ 3 cmove  ( ok )
testbuf 4 type ( aaaa ok )
\end{lstlisting}

Note the propagation in the result. \texttt{move}, however, doesn't propagate.
The last two lines would be:

\begin{lstlisting}[frame=lines]
testbuf dup char+ 3 move  ( ok )
testbuf 4 type ( aabc ok )
\end{lstlisting}

In practice, \texttt{move} is usually what you want to use.
