% Overview of Tali Forth
% Scot W. Stevenson

% -------------------------------------
\section{Design considerations}

When creating a new Forth, there are a bunch of design decisions to be
made.\footnote{The best introduction to these questions is found in
\href{http://www.bradrodriguez.com/papers/moving1.htm}{\textit{Design Decisions
in the Forth Kernel}} by Brad Rodriguez\index{Rodriguez, Brad}} Spoiler alert:
Tali Forth ended up as a subroutine-threaded variant with a 16-bit cell size and
a dictionary that keeps headers and code separate. If you don't care and just
want to use the program, skip ahead. 

\subsection{Characteristics of the 65c02}

Since this is a bare-metal Forth, the most important consideration is the target
processor. The 65c02 only has one full register, the accumulator
A\index{accumulator}, and two secondary registers X\index{X register} and
Y\index{Y register}. All are 8-bit wide. There are 256 bytes that are more
easily addressable on the Zero Page\index{Zero Page}. A single hardware stack is
used for subroutine jumps. The address bus\index{address bus} is 16 bits wide
for a maximum of 64 KiB of RAM\index{RAM} and ROM\index{ROM}. For the default,
simple setup, we assume 32 KiB of each. 

\subsection{Cell size}

The 16 bit address bus\index{address bus} suggests the cell size should be 16
bits as well. This is still easy enough to realize on a 8-bit MPU, though not as
comfortable as working with the 65816\index{65816}, the 65c02's big brother,
with an actual 16 bit register size.

\subsection{Threading technique}\index{threading}

A `thread' in Forth is simply a list of addresses of words to be executed. 
There are four basic threading techniques:\footnote{For the 8086 MPU, Guy
Kelly\index{Kelly, Guy} compared various Forth implementations in
1992\cite{kelly92}}

\begin{description}
        \item [Indirect threaded (ITC)] The oldest, original variant, used by FIG
                Forth\index{FIG Forth}. All other versions are modifications of this 
                model.
        \item [Direct threaded (DTC)] Includes more assembler code to speed
                things up, but slightly larger than ITC. 
        \item [Token threaded (TTC)] The reverse of DTC in that it is slower, but uses
                less space than the other Forths. Words are created as a table
                of tokens.
        \item [Subroutine threaded (STC)] This technique converts the words to a simple
                series of \texttt{jsr} combinations. 
\end{description}

Our lack of registers and the goal of creating a simple and easy to understand
Forth makes subroutine threading the most attractive solution. We will try to
mitigate the pain caused by the 12 cycle cost of each and every \texttt{jsr/rts}
combination by including a relatively large number of native words\index{native
words}. 

\subsection{Register use}

The lack of registers -- and 16 bit registers at that -- becomes apparent when
you realize that Forth classically uses at least four `virtual' registers:

\begin{table}[h !]
        \centering
        \label{tab:registers}
        \begin{tabular}{| c | l |}
                \hline
                W   & Working register\\
                IP  & Interpreter Pointer\\
                DSP & Data Stack Pointer\\
                RSP & Return Stack Pointer\\
                \hline
        \end{tabular}
        \caption{The classic Forth registers}
\end{table}

On a modern processor like a RISC-V\index{RISC-V} RV32I CPU with 32 registers of
32 bit each, this wouldn't be a problem. In fact, we'd be trying to figure out
what else we could keep in a register. On the 65c02, at least we get the RSP for
free with the built-in stack pointer. This still leaves three registers. We cut
that number down by one through subroutine threading, which gets rid of the IP.
For the DSP, we use the 65c02's Zero Page\index{Zero Page} indirect addressing
mode with the X register\index{X register}. This leaves W, which we put on the
Zero Page as well. 


\subsection{Data Stack design}

We'll go into greater detail on how the Data Stack works in a later chapter when
we look at the internals. Briefly, the stack is realized on the Zero
Page\index{Zero Page} for speed. For stability, we provid underflow
checks\index{underflow!detection} in the relevant words, but give the user the
option of stripping it out for native compilation\index{native compilation}.


\subsection{Dictionary structure}

Each Forth word consists of the actual code and the header\index{header} which
holds the meta-data. Part of this data is the single-linked list of words which
is searched. 

In constrast to Tali Forth 1\index{Tali Forth 1}, which kept the header and body
of the words together, Tali Forth 2 keeps them separate. This lets us play
various tricks with the code to make it more effective.

\subsection{Deeper down the rabbit hole}

This concludes our overview of the basic Tali Forth 2 structure. For those
interested, a later chapter will provide far more detail.
